\documentclass[twoside,12pt]{wipb}

\katedra{Medi�w Cyfrowych i Grafiki Komputerowej}
\typpracy{in�ynierska}
%typpracy{magisterska}
\temat{Edytor modeli 3D opartych\newline o woksele}
\autor{Pawe� Aleksiejuk}
\promotor{dr in�. �ukasz Gadomer}
\indeks{105527}
\studia{stacjonarne}
\rokakademicki{2021/2022}
\profil{studia I stopnia}
\kierunekstudiow{Informatyka}
\specjalnosc{-}
\zakres{1. Przegl�d podobnych rozwi�za� dost�pnych na rynku.\newline 2. Zdefiniowanie wymaga� stawianych wobec rozwi�zania.\newline 3. Opracowanie prostego silnika 3D.\newline 4. Stworzenie narz�dzia do edycji modelu 3D\newline 5. Testowanie stworzonego rozwi�zania.}

\hypersetup{
pdfauthor={Pawe� Aleksiejuk},
pdftitle={Praca in�ynierska},
pdfsubject={Edytor modeli 3D opartych o woksele},
pdfkeywords={praca in�ynierska jakie� inne s�owa kluczowe},
pdfpagemode=UseNone,
linkcolor=black,
citecolor=black,
urlcolor=black
} 

\setlength{\epigraphwidth}{1\textwidth}
\setlength{\parskip}{0.2em}
\usepackage{enumitem}
\usepackage{listings}
\usepackage{algorithmic}
\usepackage{subcaption}
\usepackage{pdfpages}

\begin{document}
%\maketitle

\includepdf[pages={1}]{grafika/strona-tytulowa.docx.pdf}
\includepdf[pages={1}]{grafika/Karta_Dyplomowa_2022-01-18.docx.pdf}

\chapter*{\centering{\vspace{1in}Summary}}
\addcontentsline{toc}{chapter}{Streszczenie}
 
\epigraphhead[40]{
Subject of diploma thesis

3D model editor based on voxels.}

Streszczenie pracy po angielsku.
\cleardoublepage
\includepdf[pages={1}]{grafika/oswiadczenie-o-samodzielnosci.docx.pdf}




%\biblioteka{}

\pagestyle{plain}

\setcounter{tocdepth}{1}
\tableofcontents

\chapter*{Wst�p}
\addcontentsline{toc}{chapter}{Wst�p}
Celem pracy jest stworzenie aplikacji, kt�ra pozwoli na tworzenie modeli 3D przy u�yciu wokseli. Edytor ma umo�liwi� u�ytkownikowi zaprojektowanie w�asnego modelu 3D wykorzystuj�c wbudowane mechanizmy edycji.

Zanim rozpocz��em jak�kolwiek prac� nad aplikacj�, postaniowi�em zrobi� przegl�d podobnych rozwi�za� dost�pnych na rynku, co pozwoli�o mi na okre�lenie podstawowych funkcjonalno�ci mojego edytora. Ze wzgl�du na to, �e m�j temat opiera si� w g��wnej mierze na obs�udz� w�asnego silnika 3D, kolejnym krokiem by�o stworzenie prostej jego wersji. Gdy ju
-	Zdefiniowanie wymaga� stawianych wobec rozwi�zania
-	Opracowanie prostego silnika 3D
-	Stworzenie narz�dzia do edycji modelu 3D
-	Testowanie stworzonego rozwi�zania

\chapter{Przegl�d istniej�cych rozwi�za�}

Z uwagi na specjalistyczne zastosowanie stworzonego edytora graficznego, a mianowicie tworzenie specjalnych obiekt�w obs�ugiwanych przez wbudowany silnik graficzny, istniej�ce rozwi�zania w g��wnej mierze maj� s�u�y� jako wykaz podstawowych, jak i dodatkowych funkcjonalno�ci do mo�liwej implementacji w ostatecznym rozwi�zaniu.

\section{MagicaVoxel}

MagicaVoxel \cite{magicavoxel_page} jest najpopularniejszym darmowym desktopowym edytorem wokseli dost�pnym aktualnie na rynku. Stworzony i na bie��co aktualizowany przez u�ytkownika o pseudonimie @ephtracy pozwala na nie tylko tworzenie modeli, ale te� zdj�� do p�niejszego udost�pniania. Taka funkcjonalno�� pozwala na przetestowanie modelu w r�nych warunkach, kt�re s� edytowalne poprzez parametry w wewn�trznym silniku renderuj�cym. Interfejs  rozwi�zania zosta� przedstawiony na rysunku \ref{rys1.1-magicavoxel}

\begin{figure}[htb]
\centering
\includegraphics[width=0.9\textwidth, keepaspectratio]{grafika/magicavoxel.png}
\caption{Ekran startowy programu MagicaVoxel (Windows), �r�d�o: \cite{magicavoxel_page}}
\label{rys1.1-magicavoxel}
\end{figure}

G��wne atuty oprogramowania wed�ug producenta:
\begin{itemize}
\item Zaawansowany wewn�trzny silnik renderuj�cy. 
\item Ca�kowicie darmowe oprogramowanie, nawet w przypadku u�ycia komercyjnego.
\end{itemize}

MagicaVoxel jest dost�pny za darmo na platformach Windows i macOS.

\section{Mega Voxels Play}

Mega Voxels Play \cite{mega_voxels_play_page} to darmowy mobilny edytor stworzony przez Go Real Games. Tak jak wi�kszo�� edytor�w wokselowych, pozwala na podstawowe operacje takie jak dodawanie, usuwanie i malowanie. Aplikacja posiada wbudowany sklep, kt�ry pozwala na pobranie gotowych modeli, w celu p�niejszego wykorzystania. Interfejs  rozwi�zania zosta� przedstawiony na rysunku \ref{rys1.2-mega-voxels-play}

\begin{figure}[htb]
\centering
\includegraphics[width=0.8\textwidth, height=0.4\textheight, keepaspectratio]{grafika/mega-voxels-play.png}
\caption{Ekran startowy programu Mega Voxels Play (Android), �r�d�o: \cite{mega_voxels_play_page}}
\label{rys1.2-mega-voxels-play}
\end{figure}

G��wne atuty oprogramowania wed�ug producenta:
\begin{itemize}
\item Du�a ilo�� bazowych modeli do pobrania.
\item Prosto�� w obs�udze.
\item Wsparcie dla AR (Rozszerzonej rzeczywisto�ci).
\item R�ne efekty przetwarzania ko�cowego.
\end{itemize}

Mega Voxels Play jest dost�pny za darmo na platformach mobilnych (Android i iOS).

\section{Qubicle}

Qubicle \cite{qubicle_page} jest zaawansowanym desktopowym narz�dziem stworzonym przez Minddesk, przeznaczonym do tworzenia wokselowych modeli. Z por�wnaniem do poprzednik�w, aplikacja nie posiada limitu wielko�ci modeli, co pozwala u�ytkownikom na swobodne tworzenie wielkich modeli, jak i ca�ych teren�w. Dodatkowo opr�cz standardowego w edytorach formatu .obj (Wavefront File), wspierane s� te� takie formaty jak .fbx (Autodesk), .dae (Collada). Interfejs  rozwi�zania zosta� przedstawiony na rysunku \ref{rys1.3-qubicle}

\begin{figure}[htb]
\centering
\includegraphics[width=0.9\textwidth, keepaspectratio]{grafika/qubicle.png}
\caption{Ekran startowy programu Qubicle (Windows, Steam), �r�d�o: \cite{qubicle_page}}
\label{rys1.3-qubicle}
\end{figure}

G��wne atuty oprogramowania wed�ug producenta:
\begin{itemize}
\item Bardzo du�o narz�dzi do edycji.
\item Proste w obs�udze.
\item Wbudowane narz�dzie do konwersji z modelu siatkowego na model wokselowy.
\item Wiele format�w do eksportu modeli.
\end{itemize}

Qubicle jest dost�pny w czterech wersjach na platformach Windows i macOS, wersja okrojona (demo) za darmo, wersja podstawowa (bazowa) za 53.99 PLN, wersja rozszerzona (indie) za 89.99 PLN i pe�na opcja (pro) za 410.56 PLN.

\section{Goxel}

Goxel \cite{goxel_page} jest otwartym oprogramowaniem do edycji modeli wokselowych na komputery osobiste i urz�dzenia mobilne stworzone przez u�ytkownika o pseudonimie @guillaumechereau (GitHub). G��wn� funkcjonalno�ci� Goxel, jest mo�liwo�� tworzenia warstw, w taki sam spos�b jak w popularnych aplikacjach do manipulacji obrazami, mi�dzy innymi takim jaki jest Adobe Photoshop. Interfejs  rozwi�zania zosta� przedstawiony na rysunku \ref{rys1.4-goxel}

\begin{figure}[htb]
\centering
\includegraphics[width=0.9\textwidth, keepaspectratio]{grafika/goxel.png}
\caption{Ekran Startowy programu Goxel (Windows), �r�d�o: \cite{goxel_page}}
\label{rys1.4-goxel}
\end{figure}

G��wne atuty oprogramowania wed�ug producenta:
\begin{itemize}
\item Niesko�czona wielko�� sceny.
\item Mo�liwo�� tworzenia obiekt�w na r�nych warstwach.
\item Wieloplatformowo��.
\item Wiele format�w do eksportu modeli.
\end{itemize}

Goxel jest dost�pny za darmo na platformach Windows, Linux, iOS i macOS, a w przypadku platformy Android za op�at� 25.99 PLN.

\section{VoxEdit Beta}

VoxEdit Beta \cite{voxedit_beta_page} jest darmowym oprogramowaniem stworzonym przez Pixowl do gry The Sandbox Game. Unikaln� funkcjonalno�ci� na tle innych aplikacji do edycji wokseli, jest mo�liwo�� montowania szkieletu i jego p�niejszej animacji. Interfejs  rozwi�zania zosta� przedstawiony na rysunku \ref{rys1.5-voxedit-beta}

\begin{figure}[htb]
\centering
\includegraphics[width=0.9\textwidth, keepaspectratio]{grafika/voxedit-beta.png}
\caption{Ekran startowy programu VoxEdit Beta (Windows), �r�d�o: \cite{voxedit_beta_page}}
\label{rys1.5-voxedit-beta}
\end{figure}

G��wne atuty oprogramowania wed�ug producenta:
\begin{itemize}
\item Mo�liwo�� tworzenia animacji.
\item Specjalny tryb edycji blok�w.
\item Przyjazny interfejs dla u�ytkownika.
\end{itemize}

VoxEdit Beta jest dost�pny za darmo na platformach Windows i macOS.



\chapter{Projekt systemu}

\section{W}

Moje rozwi�zanie do edycji tr�jwymiarowych obiekt�w sk�ada si� z trzech warstw:

\begin{itemize}
\item Warstwa widoku.
\item Warstwa aplikacji.
\item Warstwa obiektu.
\end{itemize}




\section{Wymagania funkcjonalne}

\begin{itemize}
\item Tworzenie modeli 3D.
\item Prosty i intuicyjny interfejs u�ytkownika.
\item Edycja modeli w czasie rzeczywistym.
\item Zapis i odczyt modelu.
\item Zmiana w�a�ciwo�ci o�wietlenia jak i materia��w pojedynczych wokseli.
\end{itemize}

\section{Wymagania niefunkcjonalne}

\begin{itemize}
\item Mo�liwo�� ponownego u�ycia silnika 3D w innych projektach.
\item Wysoka responsywno�� na zmiany w modelu (<16.6 ms).
\end{itemize}

\section{Interfejs graficzny}

\section{Diagram przypadk�w u�ycia i opisy}

\begin{figure}[htb]
\centering
\includegraphics[width=1\textwidth, keepaspectratio]{grafika/diagram-przypadkow-uzycia.png}
\caption{Diagram przypadk�w u�ycia, �r�d�o: opracowanie w�asne} 
\label{rys-diagram-przypadkow-uzycia}
\end{figure}

G��wnym narz�dziami w przypadku

\begin{table}[htb]
\centering
\caption[Opis przypadku u�ycia ,,Dodaj woksel'']{Opis przypadku u�ycia ,,Dodaj woksel''}
\begin{tabular}{|>{\bfseries}p{.3\textwidth}|p{.7\textwidth}|}
\hline
\rowcolor{Gray}
\rowstyle{\bfseries}
Sekcja & Tre�� \\\hline
Uczestnicz�cy aktorzy & U�ytkownik, Widok, Aplikacja, Obiekt \\\hline
Warunki wst�pne & W okienku ,,Edit Mode'' zaznaczony tryb ,,Add'' \\\hline
Warunki ko�cowe & Dodanie woksela do obiektu \\\hline
Rezultat & Pojawienie si� woksela w miejscu wskazanym przez u�ytkownika \\\hline
Scenariusz g��wny & 
\begin{enumerate}[topsep=0pt, leftmargin=*]
\item U�ytkownik wybiera materia� z listy, b�d� dodaje sw�j w�asny i go
zatwierdza.
\item U�ytkownik nacelowuje na interesuj�c� go sciank� woksela, w celu postawienia na obok niej nowego woksela, po czym zatwierdza prawym przyciskiem myszy.
\item Aplikacja przekazuje do obiektu dane klikni�cia.
\item Obiekt aktualizuje struktur� danych.
\item Widok zostaje od�wie�ony w nast�pnej klatce.
\item U�ytkownik widzi efekt swojego dzia�ania na modelu 3D.
\end{enumerate} \\\hline
Scenariusz wyj�tku & Zdarzenie: U�ytkownik nie klikn�� na �ciank� istniej�cego woksela 

Wynik: Brak dodania woksela do modelu 3D \\\hline
Zale�no�ci czasowe & \begin{enumerate}[topsep=0pt, leftmargin=*]
\item Cz�stotliwo�� wykonania: 0 lub wi�cej na sesj�.
\item Typowy czas realizacji: 8.9 ms.
\item Maksymalny czas realizacji: 33,2 ms. 
\end{enumerate} \\\hline
Warto�ci uzyskane przez aktor�w po zako�czeniu przypadk�w u�ycia & \begin{enumerate}[topsep=0pt, leftmargin=*]
\item Pojawienie si� woksela w miejscu i o materiale wybranym przez u�ytkownika.
\item Obiekt posiada zaktualizowan� struktur� o woksela.
\end{enumerate} \\\hline
\end{tabular}
\label{tabela_wodaj_voksel}
\end{table}

\begin{table}[htb]
\centering
\caption[Opis przypadku u�ycia ,,Usu� woksel'']{Opis przypadku u�ycia ,,Usu� woksel''}
\begin{tabular}{|>{\bfseries}p{.3\textwidth}|p{.7\textwidth}|}
\hline
\rowcolor{Gray}
\rowstyle{\bfseries}
Sekcja & Tre�� \\\hline
Uczestnicz�cy aktorzy & U�ytkownik, Widok, Aplikacja, Obiekt  \\\hline
Warunki wst�pne & W okienku ,,Edit Mode'' zaznaczony tryb ,,Remove''  \\\hline
Warunki ko�cowe & Usuni�cie wskazanego woksela z obiektu \\\hline
Rezultat & Znikni�cie woksela w miejscu wskazanym przez u�ytkownika \\\hline
Scenariusz g��wny & 
\begin{enumerate}[topsep=0pt, leftmargin=*]
\item U�ytkownik nacelowuje na interesuj�cy go woksel, po czym zatwierdza
prawym przyciskiem myszy.
\item Aplikacja przekazuje do obiektu dane klikni�cia.
\item Obiekt zwraca woksel zainteresowania.
\item Aplikacja usuwa zwr�cony woksel
\item Widok zostaje od�wie�ony w nast�pnej klatce.
\item U�ytkownik widzi efekt swojego dzia�ania na modelu 3D.
\end{enumerate} \\\hline
Scenariusz wyj�tku & Zdarzenie: U�ytkownik nie klikn�� w istniej�cego woksela 

Wynik: Brak usuni�cia woksela z modelu 3D \\\hline
Zale�no�ci czasowe & \begin{enumerate}[topsep=0pt, leftmargin=*]
\item Cz�stotliwo�� wykonania: 0 lub wi�cej na sesj�.
\item Typowy czas realizacji: 5.9 ms.
\item Maksymalny czas realizacji: 16,6 ms. 
\end{enumerate} \\\hline
Warto�ci uzyskane przez aktor�w po zako�czeniu przypadk�w u�ycia & \begin{enumerate}[topsep=0pt, leftmargin=*]
\item Znikni�cie woksela w miejscu wybranym przez u�ytkownika.
\item Obiekt posiada zaktualizowan� struktur� bez klikni�tego woksela.
\end{enumerate} \\\hline
\end{tabular}
\label{tabela_usun_woksel}
\end{table}

\chapter{Zastosowane technologie i rozwi�zania}

Aby stworzy� aplikacj�, wpierw trzeba podj�� decyzj� dotycz�c� technologii w jakiej ma ona powsta�. Programy dzia�aj�ce w czasie rzeczywistym, wymagaj� du�ej odpowiedzialno�ci ze strony programisty, by zapewni� u�ytkownikowi jak najp�ynniejsze do�wiadczenie podczas u�ytkowania. W tym celu, wybrano rozwi�zania pozwalaj�ce programi�cie na jak najwi�ksz� ingerencj� w spos�b dzia�ania, jednocze�nie pozwalaj�c na wykorzystanie gotowych rozwi�za�.

\section{J�zyki programowania}

Aplikacja zosta�a napisana w j�zyku C++ w wersji ISO/IEC 14882:2011 \cite{iso_iec_page} (C++11), z wyj�tkiem plik�w cieniowania barw (ang. \textit{shader files}), kt�re zosta�y napisane w j�zyku GLSL (OpenGL Shading Language) w wersji ,,330 core'' \cite{glsl_page}. S� wysokopoziomowymi j�zykami, ze sk�adni� pochodz�c� z j�zyka C, pozwalaj�cymi na bezpo�redni dost�p do zasob�w sprz�towych i funkcji systemowych.

\section{�rodowisko programistyczne}

Z uwagi na wymaganie multiplatformowo�ci postawione w etapie projektowania systemu, wszystkie narz�dzia programistyczne do tworzenia oprogramowania, powinny te zapotrzebowanie spe�nia�. Visual Studio Code \cite{vscode_page} jest idealnym przyk�adem narz�dzia, b�d�cego multiplatformowe, bezp�atne oraz elastyczne. 
G��wnym atutem tego IDE (ang. \textit{Integrated Development Environment}), jest ogromna biblioteka rozszerze�. Do stworzenie aplikacji, u�yte zosta�y nast�puj�ce pozycje: 
\begin{itemize}
\item C/C++ \cite{c_cpp_page} - wsparcie dla j�zyk�w C i C++.
\item CMake \cite{cmake_page} - wsparcie dla j�zyka CMake.
\item CMake Tools \cite{cmake_tools_page} - integracja programu CMake.
\item VSCode Icons \cite{vscode-icons_page} - zestaw ikon.
\item Path Intellisense \cite{path_intellisense_page} - automatyczne ko�czenie nazw plik�w.
\end{itemize}

\section{Biblioteki}

Jedn� z najwi�kszych przewag j�zyka C++ jest �atwo�� korzystania z bibliotek napisanych w C, C++ lub innych, niezale�nie od platformy systemowej. Kontynuuj�c temat multiplatformowo�ci, systemy obs�ugi okien nie jest wsp�lny dla ka�dego z system�w operacyjnych. W celu rozwi�zania tego problemu, u�yta zosta�a biblioteka GLFW \cite{glfw_page}, pe�ni�ca rol� interfejsu programistycznego aplikacji (ang. \textit{Application Programming Interface}), wspieraj�ca systemy okien takie jak Windows, macOS, X11 oraz Wayland.

Silniki graficzne s� zaawansowanymi modu�ami, kt�re wykonuj� wiele oblicze� matematycznych. W celu zapewnienia najwy�szej wydajno�ci dzia�ania tych oblicze�, jak i zapewnieniu zgodno�ci struktur matematycznych pomi�dzy kodem pisanym w C++, a kodem pisanym w GLSL, wybrano bibliotek� glm \cite{glm_page}.

Do komunikacji pomi�dzy u�ytkownikiem a silnikiem 3D, wykorzystano bibliotek� ImGui \cite{imgui_page}, pe�ni�c� rol� graficznego interfejsu u�ytkownika (ang. \textit{GUI Graphical User Interface}. Zosta�a stworzona specjalnie na potrzeby szybkich iteracji, potrzebnych mi�dzy innymi w silnikach gier, aplikacjach 3D czasu rzeczywistego oraz aplikacjach pe�noekranowych.

Zastosowanie OpenGL w aplikacji wymaga u�ycia biblioteki �aduj�cej, odpowiedzialnej za �adowanie wska�nik�w, funkcji w czasie rzeczywistym, j�der, jak i rozszerze�. \cite{opengl_loading_library_page} Jedn� z takich bibliotek jest Glad \cite{glad_page}, kt�ra jest generowana przez u�ytkownika na podstawie jego potrzeb.
\chapter{Realizacja projektu}

Kolejnym etapem jest i

\section{Struktura danych}

Klasa ,,Object'' (listing \ref{object_code}) jest odpowiedzialna za przechowywanie i obs�ug� danych dotycz�cych modelu. Jako rozwi�zanie do przetrzymywania danych w strukturze, autor zastosowa� po��czenie tablicy haszuj�cej (ang. \textit{hash table}) jako wyznacznik istnienia woksela w danym miejscu oraz wektora wokseli, zawieraj�cego struktury ,,Voxel'' (rysunek \ref{rys-struct_voxel__coll__graph}). 

\begin{lstlisting}[language={C++}, caption={Fragment kodu klasy ,,Object'' odpowiedzialnego za obs�ug� modelu 3D},label={object_code}]
class Object
{
public:
  Object();
  void Draw(MVP mvp, glm::vec3 cameraPosition, Light light);
  void AddVoxel(glm::ivec3 pos, Material mat);
  void ChangeColor(Voxel *voxel, Material mat);
  void RemoveVoxel(Voxel *voxel);
  void RemoveVoxel(glm::vec3 pos);
  void Reset();
  void Save();
  void Load(std::string objectPath);
  Voxel *CheckRay(glm::vec3 ray_origin, glm::vec3 ray_dir, glm::vec3 &newBlockLoc);
  std::vector<Voxel> GetListOfVoxels();

  std::string name;
private:
  ...
  std::vector<Voxel> m_voxels;
  bool m_hashVoxels[VOXEL_COUNT][VOXEL_COUNT][VOXEL_COUNT];
};
\end{lstlisting}

\begin{figure}[htb]
\centering
\includegraphics[width=0.5\textwidth, keepaspectratio]{grafika/struct_voxel__coll__graph.png}
\caption{Diagram struktury ,,Voxel'', �r�d�o: wygenerowane za pomoc� doxygen \cite{doxygen_page}} 
\label{rys-struct_voxel__coll__graph}
\end{figure}

Aplikacja obs�uguje dwie w�asne struktury danych, kt�re s� przechowywane, na poziomie dysku. Pozwala to u�ytkownikowi na dost�p do tych danych, nawet po zako�czeniu sesji.

Plik tekstowy z zako�czeniem \verb|.mat| odpowiedzialny jest za przechowywanie w�a�ciwo�ci materia��w. Tworzony jest za pomoc� okna ,,Material'', poprzez globaln� funkcj� ,,saveMaterial'' przedstawion� na listingu \ref{material_save_code}.

\begin{lstlisting}[language={C++}, caption={Fragment kodu funkcji zapisuj�cej w�a�ciwo�ci materia�u do pliku tekstowego},label={material_save_code}]
void saveMaterial(Material mat, const std::string &matName, bool edit)
{
	std::cout << "MATERIAL::SAVE_MATERIAL ";
	std::string matPath = std::string(FILES_PATH) + matName + MATERIAL_FILE_EXTENSION;
	std::cout << matPath << " ";
	std::ofstream file(matPath);
	if (file.bad() || file.fail())
	{
		std::cout << "FILE_BAD" << std::endl;
		return;
	}
	file << mat.name << std::endl;
	file << mat.ambient[0] << " " << mat.ambient[1] << " " << mat.ambient[2] << std::endl;
	file << mat.diffuse[0] << " " << mat.diffuse[1] << " " << mat.diffuse[2] << std::endl;
	file << mat.specular[0] << " " << mat.specular[1] << " " << mat.specular[2] << std::endl;
	file << mat.shininess;
	file.close();
	...
}
\end{lstlisting}
\section{Implementacja wybranych funkcjonalno�ci}


\subsection{Interakcja z obiektem}

Najwa�niejsz� funkcjonalno�ci� do implementacji by�a interakcja z obiektem. Przyci�ni�cie lewego przycisku myszy (\verb|GLFW_MOUSE_BUTTON_LEFT| i \verb|GLFW_PRESS|) w oknie wygenerowanym przez bibliotek� GLFW, pozwala na obliczenie kierunku promienia maj�cego pocz�tek w miejscu kamery. W celu obliczenia tego promienia, zastosowana zosta�a funkcja ,,getRayCast'' \cite{tranformation_raycast_page}, kt�rej zadaniem jest przekszta�cenie punktu 2D z przestrzeni rzutni (ang. \textit{viewport space}) do promienia 3D w przestrzeni �wiata (ang. \textit{world space}). 

\begin{lstlisting}[language={C++}, caption={Fragment kodu klasy ,,VoxelGame'' odpowiedzialnego za obs�ug� lewego przycisku myszy},label={mouse_code}]
if (button == GLFW_MOUSE_BUTTON_LEFT && action == GLFW_PRESS)
    {
      glm::vec3 ray_origin = voxelGame->camera->Position;
      glm::vec3 ray_dir = voxelGame->getRayCast(voxelGame->mvp.projection, voxelGame->mvp.view);
      glm::vec3 newBlockLoc = glm::vec3(0.f);
      Voxel *t_voxel = voxelGame->object->CheckRay(ray_origin, ray_dir, newBlockLoc);
      
      ...
    }
\end{lstlisting}

W celu okre�lenia intersekcji promienia z wokselem, w klasie ,,Object'' zaimplementowano funkcj� ,,CheckRay'' zmodyfikowan� na potrzeby tego testu przeci�cia z artyku�u ,,An Efficient and Robust Ray-Box Intersection Algorithm'' \cite{ray_box_article}. Modyfikacja ta doda�a mo�liwo�� obliczenia nie tylko pozycji woksela, ale te� i jego �cianki. W zale�no�ci od trybu edycji u�ytkownika, w przypadku intersekcji z wokselem, wykonywane s� nast�puj�ce funkcje:

\begin{itemize}
\item ,,ChangeColor'' (implementacja w sekcji \nameref{change_color_voxel_label}) s�u��ca do zmiany koloru woksela.
\item ,,RemoveVoxel'' (implementacja w sekcji \nameref{remove_voxel_label}) s�u��ca do usuni�cia woksela.
\item ,,AddVoxel'' (implementacja w sekcji \nameref{add_voxel_label}) s�u��ca do postawienia woksela na �ciance obliczonej przez ,,CheckRay''.
\end{itemize}

\begin{lstlisting}[language={C++}, caption={Dalszy fragment kodu z listingu \ref{mouse_code}},label={voxel_hit_code}]
if (t_voxel)
      {
        if (voxelGame->stateHandler->GetColorMode())
          voxelGame->object->ChangeColor(t_voxel, loadMaterial(voxelGame->activeMaterialName));
        if (voxelGame->stateHandler->GetRemoveMode())
          voxelGame->object->RemoveVoxel(t_voxel);
        if (voxelGame->stateHandler->GetAddMode())
        {
          voxelGame->object->AddVoxel(t_voxel->pos + newBlockLoc, loadMaterial(voxelGame->activeMaterialName));
        }
      }
\end{lstlisting}

\subsection{Dodanie woksela}
\label{add_voxel_label}

\begin{lstlisting}[language={C++}, caption={Fragment kodu klasy ,,Object'' odpowiedzialnego za dodawanie woksela},label={add_voxel_code}]
void Object::AddVoxel(glm::ivec3 pos, Material mat)
{
    glm::ivec3 t_pos = glm::ivec3(VOXEL_COUNT / 2 + pos.x, VOXEL_COUNT / 2 + pos.y, VOXEL_COUNT / 2 + pos.z);
    if (t_pos.x < 0 || t_pos.x > VOXEL_COUNT)
    {
        std::cout << "OBJECT::ADD_VOXEL::POS::X Out of bounds " << std::endl;
        return;
    }
    if (t_pos.y < 0 || t_pos.y > VOXEL_COUNT)
    {
        std::cout << "OBJECT::ADD_VOXEL::POS::Y Out of bounds " << std::endl;
        return;
    }
    if (t_pos.z < 0 || t_pos.z > VOXEL_COUNT)
    {
        std::cout << "OBJECT::ADD_VOXEL::POS::Z Out of bounds " << std::endl;
        return;
    }
    if (m_hashVoxels[t_pos.x][t_pos.y][t_pos.z])
    {
        std::cout << "OBJECT::ADD_VOXEL Voxel already here" << std::endl;
        return;
    }
    Voxel t_voxel;
    t_voxel.pos = pos;
    t_voxel.mat = mat;
    m_voxels.push_back(t_voxel);
    m_hashVoxels[t_pos.x][t_pos.y][t_pos.z] = true;
    std::cout << "OBJECT::ADD_VOXEL (" << t_voxel.pos.x << ", "
              << t_voxel.pos.y << ", " << t_voxel.pos.z << ") ("
              << t_voxel.mat.name << ")" << std::endl;
}
\end{lstlisting}

\subsection{Usuni�cie woksela}
\label{remove_voxel_label}

\begin{lstlisting}[language={C++}, caption={Fragment kodu klasy ,,Object'' odpowiedzialnego za usuni�cie woksela},label={remove_voxel_code}]
void Object::RemoveVoxel(Voxel *voxel)
{
    glm::ivec3 t_pos = glm::ivec3(VOXEL_COUNT / 2 + voxel->pos.x, VOXEL_COUNT / 2 + voxel->pos.y, VOXEL_COUNT / 2 + voxel->pos.z);
    m_hashVoxels[t_pos.x][t_pos.y][t_pos.z] = false;
    m_voxels.erase(m_voxels.begin() + (voxel - &m_voxels.front()));
}
\end{lstlisting}

\subsection{Zmiana materia�u woksela}
\label{change_color_voxel_label}

\begin{lstlisting}[language={C++}, caption={Fragment kodu klasy ,,Object'' odpowiedzialnego za zmian� materia�u woksela},label={change_color_voxel_code}]
void Object::ChangeColor(Voxel *voxel, Material mat)
{
    voxel->mat = mat;
}
\end{lstlisting}

\section{Testy aplikacji}
\chapter{Dokumentacja techniczna}

Dokumentacja techniczna aplikacji stanowi wa�ny element u�ytkownika do nawigacji po systemie. Zapewnia ona wprowadzenie do wygl�du aplikacji, jak i opisuje w jaki spos�b osi�gn�� po��dany efekt w aplikacji.

\section{Prezentacja systemu}
\label{prezentacja_systemu}
G��wnym za�o�eniem interfejsu by�a prostota i customizowalno��. Osi�gni�te to zosta�o poprzez wyeksponowanie modelu, kt�ry jest renderowany w czasie rzeczywistym przez silnik 3D edytora. 

\begin{figure}[htb]
\centering
\includegraphics[width=0.9\textwidth, keepaspectratio]{grafika/voxel-editor-main.png}
\caption{Ekran startowy programu, �r�d�o: opracowanie w�asne} 
\label{rys-voxel-editor-main}
\end{figure}

\subsection{Okno ,,Scene''}

By u�ytkownikowi da� wi�cej mo�liwo�ci ustawienia wygl�du wyj�ciowego, dodano opcj� zmiany warto�ci o�wietlenia, jak i t�a aplikacji. Wygl�d okna ,,Scene'' zosta� ukazany na rysunku \ref{rys-voxel-editor-scene-edit}.

\begin{figure}[htb]
\centering
\includegraphics[width=0.6\textwidth, keepaspectratio]{grafika/voxel-editor-scene-edit.png}
\caption{Okno zmiany ustawie� sceny, �r�d�o: opracowanie w�asne} 
\label{rys-voxel-editor-scene-edit}
\end{figure}

\subsection{Okno ,,Material''}

Zmiana aktywnego materia�u odbywa si� poprzez wyb�r z listy dost�pnych, jak i stworzonych materia��w w oknie ,,Material'' (rysunek \ref{rys-voxel-editor-material-edit}).

\begin{figure}[htb]
\centering
\includegraphics[width=0.6\textwidth, keepaspectratio]{grafika/voxel-editor-material-edit.png}
\caption{Okno zmiany bie��cego materia�u, jak i jego edycji, �r�d�o: opracowanie w�asne} 
\label{rys-voxel-editor-material-edit}
\end{figure}

\subsection{Podgl�d wybranego koloru}

W celu u�atwienia wyboru koloru podczas tworzenia materia�u oraz wyboru t�a sceny, parametry ,,Color'' w przypadku okna ,,Scene'', jak i ,,Ambient'', ,,Diffuse'' i ,,Specular'' w przypadku okna ,,Material'' posiadaj� opcj� wyboru r�cznego z kwadratu kolor�w (rysunek \ref{rys-voxel-editor-rgb}).

\begin{figure}[htb]
\centering
\includegraphics[width=0.5\textwidth, keepaspectratio]{grafika/voxel-editor-rgb.png}
\caption{Okno wyboru warto�ci koloru wraz z podgl�dem po prawej stronie, �r�d�o: opracowanie w�asne} 
\label{rys-voxel-editor-rgb}
\end{figure}

\subsection{Okno ,,Debug''}

Okno ,,Debug'' (rysunek \ref{rys-voxel-editor-debug-edit}) powsta�o specjalnie z uwag� na wymagania czasowe dotycz�ce g��wnych funkcjonalno�ci. Przycisk ,,WireFrame mode'' prze��cza w silniku 3D spos�b renderowania obiekt�w na tylko kraw�dzie, za� przycisk ,,Optimized mode'' pozwala na wy��czenie zoptymalizowanego algorytmu rysowania modelu. Pod przyciskami w czasie rzeczywistym kre�lony jest wykres czasu renderowania pojedynczej klatki (ang. \textit{frame time}).

\begin{figure}[htb]
\centering
\includegraphics[width=0.6\textwidth, keepaspectratio]{grafika/voxel-editor-debug-edit.png}
\caption{Okno z informacjami o dzia�aniu silnika graficznego, �r�d�o: opracowanie w�asne} 
\label{rys-voxel-editor-debug-edit}
\end{figure}

\subsection{Zarz�dzanie oknami}

Przedstawione powy�ej okna aplikacji, s� dost�pne z poziomu paska nawigacji (rysunek \ref{rys-voxel-editor-window-options}) za pomoc� kt�rego mo�na je pokazywa� lub chowa�. Dla ka�dego z tych okienek, u�ytkownik mo�e zmieni� ich pozycj�, wielko��, jak i zminimalizowa� (rysunek \ref{rys-voxel-editor-minimized}), wed�ug w�asnego uznania. Na rysunku \ref{rys-voxel-editor-small} przedstawione zosta�o zmniejszone okno ,,Material'' z rysunku \ref{rys-voxel-editor-material-edit}. 

\begin{figure}[htb]
\centering
\includegraphics[width=0.6\textwidth, keepaspectratio]{grafika/voxel-editor-window-options.png}
\caption{Pasek nawigacyjny z wybran� opcj� ,,Window'', �r�d�o: opracowanie w�asne} 
\label{rys-voxel-editor-window-options}
\end{figure}

\begin{figure}[htb]
\centering
\includegraphics[width=0.5\textwidth, keepaspectratio]{grafika/voxel-editor-small.png}
\caption{Okno ,,Material'' ze zmienionym rozmiarem przez u�ytkownika, �r�d�o: opracowanie w�asne} 
\label{rys-voxel-editor-small}
\end{figure}

\begin{figure}[htb]
\centering
\includegraphics[width=0.5\textwidth, keepaspectratio]{grafika/voxel-editor-minimized.png}
\caption{Okno ,,Material'' w wersji zminimalizowanej, �r�d�o: opracowanie w�asne} 
\label{rys-voxel-editor-minimized}
\end{figure}

\subsection{Okno ,,Edit Modes''}

G��wnym oknem do zmiany tryb�w interakcji z modelem 3D jest ,,Edit Modes'' ukazany na rysunku \ref{rys-voxel-editor-edit-mode}. S�u�y ono do zmiany g��wnych tryb�w edycji na modelu 3D. Przez zaznaczenie jednej opcji z ,,Add'', ,,Remove'' lub ,,Color'', klikni�cie na istniej�cy model na ekranie b�dzie mia� inny rezultat.

\begin{figure}[htb]
\centering
\includegraphics[width=0.5\textwidth, keepaspectratio]{grafika/voxel-editor-edit-mode.png}
\caption{Okno trybu edycji modelu 3D, �r�d�o: opracowanie w�asne} 
\label{rys-voxel-editor-edit-mode}
\end{figure}

\subsection{Okna manipulacji plikami}

Pasek z rysunku \ref{rys-voxel-editor-window-options} posiada jeszcze jedn� opcj�, a mianowicie ,,File'', odpowiedzialn� za wczytywanie modeli z pliku, zapisywanie ich oraz tworzenie nowych. Opcj� tej zak�adki s� przedstawione na rysunku \ref{rys-voxel-editor-file-options}

\begin{figure}[htb]
\centering
\includegraphics[width=0.6\textwidth, keepaspectratio]{grafika/voxel-editor-file-options.png}
\caption{Pasek nawigacyjny z wybran� opcj� ,,File'', �r�d�o: opracowanie w�asne} 
\label{rys-voxel-editor-file-options}
\end{figure}

W przypadku opcji ,,Open Model'' przedstawionej na rysunku \ref{rys-voxel-editor-open} oraz ,,Save As'' na rysunku \ref{rys-voxel-editor-save-as}, u�ytkownik wprowadza �cie�k� wzgl�dn� do pliku w celu jego wczytania, b�d� w przypadku opcji drugiej, jego nazw�. Opcja ,,New Model'' powoduje usuni�cie wszelkich post�p�w pracy nad aktualnym obiektem, by przywr�ci� stan pocz�tkowych modelu 3D. Operacja ta jest niemo�liwa do cofni�cia.

\begin{figure}[htb]
\centering
\includegraphics[width=0.5\textwidth, keepaspectratio]{grafika/voxel-editor-open.png}
\caption{Okno wczytania modeli do edytora, �r�d�o: opracowanie w�asne} 
\label{rys-voxel-editor-open}
\end{figure}

\begin{figure}[htb]
\centering
\includegraphics[width=0.5\textwidth, keepaspectratio]{grafika/voxel-editor-save-as.png}
\caption{Okno zapisu aktualnego modelu 3D do pliku, �r�d�o: opracowanie w�asne} 
\label{rys-voxel-editor-save-as}
\end{figure}

\subsection{Konsola}

W tymi miejscu, znajduj� si� wszystkie informacje na temat czynno�ci u�ytkownika w aplikacji. G��wnym za�o�eniem konsoli jest mo�liwo�� okre�lenia przyczyny krytycznego wy��czenia edytora. Narz�dzie te jest uruchamiane i wy��czane wraz z edytorem.

\begin{figure}[htb]
\centering
\includegraphics[width=0.9\textwidth, keepaspectratio]{grafika/voxel-editor-console.png}
\caption{Konsola aplikacji, �r�d�o: opracowanie w�asne} 
\label{rys-voxel-editor-console}
\end{figure}

\section{Instrukcja u�ytkownika}

W pe�ni zbudowana aplikacja w wersji przygotowanej w ramach niniejszej pracy jest dost�pna pod adresem \url{https://github.com/DuDuSteo/VoxelGameEngine/releases/tag/v1.0.0} \cite{app_page}. Po pobraniu pliku \verb|VoxelEditor_v1.0.0_WIN_64.zip|, wystarczy tylko wyodr�bni� zawarto��. Edytor widnieje pod nazw� \verb|VoxelEditor.exe| w przypadku system�w \verb|Windows|. Aplikacja przyj�a standardowy w�r�d edytor�w graficznych model poruszania si� kamer�:
\begin{itemize}
\item \texttt{�rodkowy przycisk myszy + Ruch kursorem} - zmiana k�ta kamery.
\item \texttt{�rodkowy przycisk myszy + Shift + Ruch kursorem} - zmiana pozycji kamery.
\end{itemize} 
Interakcja z modelem odbywa si� poprzez \texttt{lewy przycisk myszy}.

\subsection{Uruchomienie programu}

Aby uruchomi� program, nale�y dwukrotnie nacisn�� lewym przyciskiem myszy na \verb|VoxelEditor.exe|. Po uruchomieniu programu u�ytkownik zostaje przywitany ekranem startowym aplikacji przedstawionym na rysunku \ref{rys-voxel-editor-with-console}. Wszystkie ukryte okna, przedstawione w podrozdziale ,,\nameref{prezentacja_systemu}'' znajduj� si� w \verb|Pasek narz�dziowy -> Window|.

\begin{figure}[htb]
\centering
\includegraphics[width=\textwidth, keepaspectratio]{grafika/voxel-editor-with-console.png}
\caption{Edytor modeli po uruchomieniu wraz z konsol�, �r�d�o: opracowanie w�asne} 
\label{rys-voxel-editor-with-console}
\end{figure}

\subsection{Dodanie nowego materia�u}

Dodanie nowego materia�u do listy materia��w w aplikacji, odbywa si� poprzez okno ,,Material''. W podstawowym widoku edytora, by otworzy� okno materia��w, nale�y pos�u�y� si� paskiem narz�dziowym, wybieraj�c w nim \verb|Window -> Material| jak na rysunku \ref{rys-inst-select-material}.  

\begin{figure}[htb]
\centering
\includegraphics[width=0.4\textwidth, keepaspectratio]{grafika/inst-select-material.png}
\caption{Wyb�r okna ,,Material'' z paska narz�dziowego, �r�d�o: opracowanie w�asne}
\label{rys-inst-select-material}
\end{figure}

Specyfikacja materia��w opiera si� na implementacji w silniku cieniowania Phonga (ang. \textit{Phong shading}) \cite{phong_implementation}. Po wyborze nazwy, u�ytkownik wype�nia danymi pola odpowiedzialne za odbicie �wiat�a otoczenia (ang. \textit{ambient}), rozproszonego (ang. \textit{diffuse}) oraz odbijanego zwierciadlanie (ang. \textit{specular}) (rysunek \ref{rys-inst-material-window}). Mo�liwe jest te� wybranie koloru z okna kolor�w, kt�re jest widoczne po klikni�ciu w kwadrat przedstawiony na rysunku \ref{rys-inst-material-window-color}.

\begin{figure}[htb]
\centering
\includegraphics[width=0.6\textwidth, keepaspectratio]{grafika/inst-material-window.png}
\caption{Prezentacja w�a�ciwo�ci materia�u w oknie ,,Material'', �r�d�o: opracowanie w�asne}
\label{rys-inst-material-window}
\end{figure}

\begin{figure}[htb]
\centering
\includegraphics[width=0.7\textwidth, keepaspectratio]{grafika/inst-material-window-color.png}
\caption{Okno wyboru koloru z kwadratu kolor�w, �r�d�o: opracowanie w�asne}
\label{rys-inst-material-window-color}
\end{figure}


\subsection{Dodanie woksela}

Aby postawi� woksel, u�ytkownik musi mie� w��czony tryb ,,Add'' w oknie ,,Edit Modes'' i wtedy naje�d�a kursorem na interesuj�c� go �ciank� (rysunek \ref{rys-inst-material-click}), w celu postawienia woksela. Nowo utworzony woksel, b�dzie posiada� wszystkie w�a�ciwo�ci aktywnego materia�u jak na rysunku \ref{rys-inst-add-voxel}.

\begin{figure}[htb]
\centering
\includegraphics[width=0.7\textwidth, keepaspectratio]{grafika/inst-material-click.png}
\caption{Klikni�cie na wybran� �ciank�, �r�d�o: opracowanie w�asne}
\label{rys-inst-material-click}
\end{figure}

\begin{figure}[htb]
\centering
\includegraphics[width=0.7\textwidth, keepaspectratio]{grafika/inst-add-voxel.png}
\caption{Wynik interakcji z opcj� ,,Add'', �r�d�o: opracowanie w�asne}
\label{rys-inst-add-voxel}
\end{figure}

\subsection{Usuni�cie woksela}

Zaznaczaj�c opcj� ,,Remove'' w oknie ,,Edit Modes'' (rysunek \ref{rys-inst-remove}), wszelkie interakcje z modelem przez u�ytkownika spowoduj� usuni�cie woksela w miejscu klikni�cia. Proces operacji przedstawiony zosta� na rysunkach \ref{rys-inst-remove-voxel} i \ref{rys-inst-removed-voxel}.

\begin{figure}[htb]
\centering
\includegraphics[width=0.45\textwidth, keepaspectratio]{grafika/inst-remove.png}
\caption{Klikni�cie na wybran� �ciank�, �r�d�o: opracowanie w�asne}
\label{rys-inst-remove}
\vspace{0.5cm}
\includegraphics[width=0.65\textwidth, keepaspectratio]{grafika/inst-remove-voxel.png}
\caption{Klikni�cie na wybranego woksela w celu usuni�cia, �r�d�o: opracowanie w�asne}
\label{rys-inst-remove-voxel}
\end{figure}

\begin{figure}[htb]
\centering
\includegraphics[width=0.65\textwidth, keepaspectratio]{grafika/inst-removed-voxel.png}
\caption{Wynik interakcji z opcj� ,,Remove'', �r�d�o: opracowanie w�asne}
\label{rys-inst-removed-voxel}
\end{figure}

\subsection{Zmiana materia�u woksela}

Ostatni� opcj� w oknie ,,Edit Modes'' jest opcja ,,Color'' odpowiadaj�ca za ,,przemalowanie'' klikni�tego woksela na aktywny materia�. Efekt operacji zmiany materia�u zosta� pokazany na rysunku \ref{rys-inst-color} i \ref{rys-inst-colored}.


\begin{figure}[htb]
\centering
\includegraphics[width=0.6\textwidth, keepaspectratio]{grafika/inst-color.png}
\caption{Klikni�cie na wybranego woksela w celu zmiany materia�u, �r�d�o: opracowanie w�asne}
\label{rys-inst-color}
\vspace{0.5cm}
\includegraphics[width=0.6\textwidth, keepaspectratio]{grafika/inst-colored.png}
\caption{Wynik interakcji z opcj� ,,Color'', �r�d�o: opracowanie w�asne}
\label{rys-inst-colored}
\end{figure}

\subsection{Zmiana o�wietlenia}

Do zmiany ustawie� o�wietlenia s�u�y okno ,,Scene'' dost�pne z poziomu paska narz�dzi \verb|Window -> Scene| (rysunek \ref{rys-inst-window-scene}). Nale�y r�wnie� podkre�li�, �e OpenGL korzysta z prawor�cznego systemu uk�adu wsp�rz�dnych, co oznacza �e dodatnia o� z jest skierowana w naszym kierunku \cite{learn_opengl}. 


\begin{figure}[htb]
\centering
\includegraphics[width=0.5\textwidth, keepaspectratio]{grafika/inst-window-scene.png}
\caption{Wyb�r okna ,,Scene'' z paska narz�dziowego, �r�d�o: opracowanie w�asne}
\label{rys-inst-window-scene}
\end{figure}

\begin{figure}[htb]
\centering
\includegraphics[width=0.5\textwidth, keepaspectratio]{grafika/coordinate_systems_right_handed.png}
\caption{Przyk�ad prawor�cznego systemu uk�adu wsp�rz�dnych, �r�d�o: \cite{learn_opengl}}
\label{rys-coordinate_systems_right_handed}
\end{figure}

Na rysunku \ref{rys-inst-scene-color} przedstawiono efekt o�wietlenia, posiadaj�cego kierunek �wiat�a o tr�jwymiarowym wektorze znormalizowanym skierowanym r�wnolegle do k�ta patrzenia kamery (x: 0, y: 0, z: -1), jak i parametrach ustawionych na kolor bia�y dla �wiat�a otoczenia (ang. \textit{ambient}), rozproszonego (ang. \textit{diffuse}) oraz odbijanego zwierciadlanie (ang. \textit{specular}). Zmiany na parametrach w oknie ,,Scene'' aktualizowane s� na silniku 3D w czasie rzeczywistym.

\begin{figure}
\centering
\includegraphics[width=0.7\textwidth, keepaspectratio]{grafika/inst-scene-color.png}
\caption{Przyk�adowe ustawienia o�wietlenia przy u�yciu okna ,,Scene'', �r�d�o: opracowanie w�asne}
\label{rys-inst-scene-color}
\end{figure}

\subsection{Zapis i odczyt}

Zapis i odczyt odbywa si� poprzez okna z listy rozwijanej ,,File'' w pasku narz�dziowym. W celu zapisania efekt�w pracy na modelem 3D, nale�y wybra� w pasku narz�dziowym \verb|File -> Save As|. Po klikni�ciu ,,Save As'', w aplikacji pojawi si� okno odpowiedzialne za zapis pliku (rysunek \ref{rys-inst-save-as}). U�ytkownik wpisuje wybran� nazw� pliku w polu tekstowym i zatwierdza przyciskiem ,,Save''. Stworzony plik znajduje si� w folderze \texttt{./files/} z rozszerzeniem \verb|.vxl|.

\begin{figure}[htb]
\centering
\includegraphics[width=0.5\textwidth, keepaspectratio]{grafika/inst-save-as.png}
\caption{Okno wyboru nazwy pliku do zapisu, �r�d�o: opracowanie w�asne}
\label{rys-inst-save-as}
\end{figure}

Analogicznie odczyt odbywa si� poprzez otworzenie okna ,,Open Model'', znajduj�cego si� \verb|File -> Open Model|, jak na przedstawionym rysunku \ref{rys-inst-load}. W przypadku operacji wczytania z pliku, u�ytkownik przekazuj� w polu tekstowym �cie�k� i rozszerzenie, ani�eli nazw� pliku.

\begin{figure}[htb]
\centering
\includegraphics[width=0.5\textwidth, keepaspectratio]{grafika/inst-load.png}
\caption{Okno wyboru �cie�ki pliku do odczytu, �r�d�o: opracowanie w�asne}
\label{rys-inst-load}
\end{figure}

\subsection{Tworzenie nowego modelu 3D}

Aby stworzy� nowy model, b�d� usun�� aktualne efekty pracy w edytorze, nale�y wybra� w pasku narz�dziowym \verb|File -> New Model| przedstawion� na rysunku \ref{rys-inst-new}. Klikni�cie opcji ,,New Model'' usunie wszystkie aktualne post�py pracy i przywr�ci model 3D do stanu pocz�tkowego.

\begin{figure}[htb]
\centering
\includegraphics[width=0.5\textwidth, keepaspectratio]{grafika/inst-new.png}
\caption{Przycisk stworzenia nowego modelu 3D w pasku narz�dziowym, �r�d�o: opracowanie w�asne}
\label{rys-inst-new}
\end{figure}

\chapter*{Podsumowanie}
\addcontentsline{toc}{chapter}{Podsumowanie}

Tematem niniejszej pracy by�o stworzenie aplikacji pozwalaj�cej na kreacj� modeli 3D opartych o woksele. Cz�� teoretyczna opiera�a si� na zapoznaniu z podstawowymi funkcjonalno�ciami istniej�cych ju� rozwi�za�, jak i dzia�aniem tych aplikacji graficznych, na podstawie kt�rych powsta� projektu systemu. 

W ramach pracy powsta� prosty silnik 3D posiadaj�cy podstawowe funkcjonalno�ci potrzebne do edycji model�w 3D, takie jak dodawanie nowych obiekt�w, usuwanie ich, oraz zmiana ich w�a�ciwo�ci. Dost�p do tych funkcjonalno�ci odbywa si� poprzez interfejs okienkowy,  pe�ni�cy role narz�dzia do edycji modelu. W ten spos�b powsta�a aplikacja pozwoli�a na osi�gni�cie w pe�ni celu niniejszej pracy, kt�rym by�o stworzenie edytora modeli 3D opartych o woksele.

Mimo, �e aplikacja pozwala na tworzenie i edycj� modeli 3D, spe�niaj�c jednocze�nie cel pracy, nie mo�na wykluczy� dalszego rozwoju tego rozwi�zania. Z uwagi na u�ycie silnika graficznego, mo�liwe jest ponowne wykorzystanie silnika w celu stworzenia gry komputerowej. Pozwoli to na jednoczesne posiadanie narz�dzia do tworzenia modeli, operuj�ce na tym samym silniku graficznym, co gra komputerowa, pozwalaj�c na �atwe  tworzenie obiekt�w kompatybilnych z gr�. Mo�liwe jest te� rozszerzenie funkcjonalno�ci samego edytora, dodaj�c po��dane rozwi�zania takie jak mo�liwo�� cofni�cia lub przywr�cenia ostatniej czynno�ci, czy mo�liwo�� wyeksportowania wi�kszej ilo�ci parametr�w silnika graficznego.





\nocite{*}
\bibliographystyle{plain}
\bibliography{bibliografia}
\addcontentsline{toc}{chapter}{Bibliografia}

% \listoffigures
{%
    \let\oldnumberline\numberline%
    \renewcommand{\numberline}{\tablename~\oldnumberline}%
    \listoftables
}
\addcontentsline{toc}{chapter}{Spis tabel}
% \listoffigures
{
    \let\oldnumberline\numberline%
    \renewcommand{\numberline}{\figurename~\oldnumberline}%
    \listoffigures
}
\addcontentsline{toc}{chapter}{Spis rysunk�w}
\lstlistoflistings
\addcontentsline{toc}{chapter}{Spis listing�w}
\raggedbottom
\listofalgorithms
\addcontentsline{toc}{chapter}{Spis algorytm�w}

\end{document}
