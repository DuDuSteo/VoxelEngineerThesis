\chapter*{Wst�p}
\addcontentsline{toc}{chapter}{Wst�p}

G��wnym celem pracy by�o stworzenie aplikacji, kt�ra pozwoli na kreacj� modeli 3D opartych o woksele. Edytor mia� na celu umo�liwi� u�ytkownikowi zaprojektowanie w�asnego modelu 3D wykorzystuj�c wbudowane mechanizmy edycji.

Motywacj� do napisania tej pracy by�o ch�� stworzenia prostego funkcjonalnego silnika graficznego wraz z narz�dziem to tworzenia modeli obs�ugiwanych przez ten silnik. W p�niejszym czasie, planuj� rozszerzy� ten projekt, tworz�c w pe�ni funkcjonaln� gr� 3D.

Zakres pracy obejmowa�:
\begin{itemize}
\item Przegl�d podobnych rozwi�za� dost�pnych na rynku
\item Zdefiniowanie wymaga� stawianych wobec rozwi�zania
\item Opracowanie prostego silnika 3D
\item Stworzenie narz�dzia do edycji modelu 3D
\item Testowanie stworzonego rozwi�zania
\end{itemize}

Rozdzia� 1 ("Przegl�d istniej�cych rozwi�za�") przedstawia 5 istniej�cych ju� na rynku edytor�w graficznych opartych o woksele, w celu zaznajomienia si� z podstawowymi funkcjonalno�ciami postawionymi przez ich autor�w. 

Rozdzia� 2 ("")
 


